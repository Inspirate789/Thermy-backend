\section*{Реферат}

Расчетно-пояснительная записка 77 с., \totalfigures\ рис., \totaltables\ табл., 41 ист., 7 прил.

Ключевые слова: научно-технические тексты, многокомпонентный термин, извлечение терминов, базы данных, PostgreSQL, NoSQL, Redis, InfluxDB, REST API, кеширование.

% Объектом исследования является модель представления данных в системе извлечения многокомпонентных терминов и их переводных эквивалентов из параллельных научно-технических текстов.

Цель работы: разработка системы извлечения многокомпонентных терминов и их переводных эквивалентов из параллельных научно-технических текстов.

Разработанное ПО является приложением для выделения терминов из текстов, их сохранения и анализа. Приложение может использоваться лингвистами для сбора терминологических баз данных и проведения исследований.

В работе было проведено исследование, которое показало, что применение кеширования может повысить производительность приложения более чем в 2 раза.

% Результаты: среди рассмотренных операционных систем были выделены:

% \begin{itemize}
	% \item Azure Sphere, Windows 10 IoT и Amazon FreeRTOS как наиболее функциональные и масштабируемые;
	% \item ОСРВ МАКС и KasperskyOS как наиболее доступные с точки зрения использования прикладных служб;
	% \item Ubuntu Core и Raspbian как наиболее адаптированные для бытового применения.
% \end{itemize}

\pagebreak
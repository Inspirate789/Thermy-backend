\section{Аналитическая часть}

% В данном разделе будут представлены основные определения, а также будет описана архитектура интернета вещей и поставлена проблема выбора операционной системы для устройств интернета вещей.

\subsection{Метод извлечения многокомпонентных терминов на основе структурных моделей}

Предлагаемый метод автоматического извлечения русскоязычных многокомпонентных терминов на основе базы данных структурных моделей терминологических словосочетаний состоит из пяти основных этапов \cite{My_article_2021}.

\begin{enumerate}[label*=\arabic*.]
	\item Анализ предложения по частям речи.
	\item Удаление частей речи и их сочетаний, которые не входят в состав терминологических словосочетаний:
	
	\begin{itemize}[label*=---]
		\item глаголы;
		\item союзы;
		\item местоимения;
		\item частицы;
		\item знаки препинания;
		\item «наречие + предлог».
		
	\end{itemize}
	
	\item Удаление из оставшихся терминов-кандидатов на наличие стоп-слов, указанных в специальной зоне словаря, и если они есть, убираем их. Под стоп-словами понимаются слова, которые образуют широко используемые коллокации с терминами, но в совокупности не являются терминами по сути, например современная химия, рассматриваемый метод синтеза о-гликозидов.
	\item Соотнесение полученных цепочек слов с шаблонами терминологических словосочетаний, которые хранятся в базе структурных моделей терминов.
	\item Проверка полученных терминов-кандидатов по словарю корпуса.
	
	\begin{enumerate}[label*=\arabic*.]
		
	\item Если термин-кандидат есть в словаре, то он извлекается как термин.
	\item Если полученный термин-кандидат отсутствует в словаре, то он отправляется терминологу для ручной обработки.
	\item Если термин-кандидат состоит из нескольких слов, то производится попытка разбить его на несколько терминов.
	
	\end{enumerate}
	
\end{enumerate}



\subsection{Постановка задачи}

Диаграмма, оформленная в соответствии с нотацией IDEF0 и отражающая декомпозицию алгоритма работы системы извлечения многокомпонентных терминов, представлена в приложении А. Исходя из специфики решаемой задачи, можно сформулировать требования к приложению \cite{My_article_2022}.


\begin{enumerate}[label*=\arabic*.]
	\item Задание исходных текстов должно производиться из файла или через специально предусмотренные поля ввода.
	\item Пользователю должна быть предоставлена возможность перед сохранением терминов в базу данных выполнять над ними следующие операции.
	
	\begin{enumerate}[label*=\arabic*.]
		\item Редактирование.
		\item Сопоставление переводных эквивалентов.
		\item Присвоение характеристик.
		
	\end{enumerate}

	\item Пользователь может анализировать и редактировать добавленные в базу данных термины путём их поиска по заданным характеристикам.
	
\end{enumerate}



\subsection{Формализация ролей}

Для работы с системой обязательным этапом является прохождение аутентификации. Пользователь может работать в системе под одной из следующих ролей.

\begin{enumerate}[label*=\arabic*.]
	\item Студент --- пользователь, имеющий возможность обрабатывать тексты, сохранять выделенные термины, а также анализировать и редактировать только те термины, которые он выделил.
	\item Преподаватель --- пользователь, обладающий функционалом, доступным роли "Студент", а также имеющий возможность анализировать и редактировать все термины, хранящиеся в базе данных.
	\item Администратор --- пользователь, имеющий возможности, доступные роли "Преподаватель", а также имеющий в распоряжении специальные функции для анализа состояния системы и настройки её работы. Также администратор имеет возможность создавать аккаунты, соответствующие любой из ролей в системе.
	
\end{enumerate}

В ходе использования приложения предусмотрена возможность смены ролей путём прохождения повторной аутентификации.

Также стоит отметить, что до входа в аккаунт пользователь считается неавторизованным и не имеет доступа к функционалу системы, так как любая работа с терминами должна быть персонализирована.

На рисунках ?????-????? представлена диаграмма вариантов использования системы в соответствии с выделенными типами пользователей.



\subsection{Формализация данных}

С учётом выделенных структур данных и типов пользователей разрабатываемая база данных должна хранить информацию о следующих сущностях:



\subsection{Анализ моделей БД}

Определение БД и СУБД

\subsubsection{Классификация СУБД по модели данных}

\textbf{Модель данных} [ссылка на новую книгу Дейта] – это абстрактное, независимое, логическое определение структур
данных, операторов над данными и прочего, что в совокупности составляет абстрактную систему, с которой взаимодействует пользователь.

По модели данных СУБД можно разделить на следующие типы.



\begin{enumerate}[label*=\arabic*.]
	\item \textbf{Дореляционные}. \newline
	[ссылка на старую книгу Дейта]
	старые (дореляционные) 
	системы можно разделить на три большие категории
	: системы с инвертированными списками (inverted list), иерархические (hierarchic) и сетевые (network). Примечание. Термин
		сетевая система в данном случае не имеет ничего общего с коммуникационной сетью, как
		описано в следующей главе.) В настоящей книге эти категории подробно не рассматриваются, поскольку, по крайней мере, с точки зрения технологии, их можно считать устаревшими.
		
	\begin{enumerate}[label*=\arabic*.]
		
		\item \textbf{Инвертированные списки (файлы)}. \newline
		БД на основе инвертированных списков представляет собой совокупность файлов, содержащих записи (таблиц). Для записей в файле определен некоторый порядок, диктуемый физической организацией данных. Для каждого файла может быть определено произвольное число других упорядочений на основании значений некоторых полей записей (инвертированных списков). Обычно для этого используются индексы. В такой модели данных отсутствуют ограничения целостности как таковые. Все ограничения на возможные экземпляры БД задаются теми программами, которые работают с БД. Одно из немногих ограничений, которое все-таки может присутствовать - это ограничение, задаваемое уникальным индексом. 
		
		\item \textbf{Иерархичекие}. \newline
		Иерархическая модель БД состоит из объектов с указателями от родительских объектов к дочерним, соединяя вместе связанную информацию. Иерархические БД могут быть представлены в виде дерева. Производительность в значительной степени зависит от подхода, выбранного самим пользователем (прикладным программистом и/или администратором базы данных).
		
		\item \textbf{Сетевые}. \newline
		К основным понятиям сетевой модели БД относятся: элемент (узел), связь. Узел — это совокупность атрибутов данных, описывающих некоторый объект. Сетевые БД могут быть представлены в виде графа. В сетевой БД логика процедуры выборки данных зависит от физической организации этих данных. Поэтому эта модель не является полностью независимой от приложения. Другими словами, если необходимо изменить структуру данных, то нужно изменить и приложение.
		
	\end{enumerate}
	
	\item \textbf{Реляционные}. \newline
	Ключевые особенности реляционной модели данных [ссылка на старую книгу Дейта]:
	■ Во-первых, отметим, что определение реляционной системы требует, чтобы база
	данных только воспринималась пользователем как набор таблиц. Таблицы в реляционной системе являются логическими, а не физическими структурами. На самом
	деле, на физическом уровне система может использовать любую из существующих
	структур памяти (последовательный файл, индексирование, хэширование, цепочку указателей, сжатие и т.п.), лишь бы существовала возможность отображать
	эти структуры в виде таблиц на логическом уровне. Данное положение можно
	сформулировать и по-другому: таблицы представляют собой абстракцию способа
	физического хранения данных, в которой все нюансы реализации на уровне физической памяти (размещение хранимых записей, упорядочение хранимых записей,
	кодировка хранимых данных, префиксы хранимых записей, хранимые структуры
	доступа, такие как индексы и т.д.) скрыты от пользователя.
	■ Во-вторых, у реляционных баз данных есть одно замечательное свойство, опреде
	ляемое так называемым информационным принципом: все информационное наполне
	ние базы данных представлено одним и только одним способом, а именно — явным за
	данием значений, помещенных в позиции столбцов в строках таблицы. Этот метод
	представления — единственно возможный для реляционных баз данных (естест
	венно, на логическом уровне). В частности, нет никаких указателей, связывающих
	одну таблицу с другой.
	
	В первом приближении
	реляционная модель состоит из следующих пяти компонентов [ссылка на старую книгу Дейта].
	1. Неограниченный набор скалярных типов (включая, в частности, логический тип
	или истинностное значение).
	2. Генератор типов отношений и соответствующая интерпретация для сгенерирован
	ных типов отношений.
	3. Возможность определения переменных отношения для указанных сгенерированных
	типов отношений.
	4. Операция реляционного присваивания для присваивания реляционных значений
	указанным переменным отношения.
	5. Неограниченный набор общих реляционных операторов (реляционная алгебра) для
		получения значений отношений из других значений отношений.
	
	Вполне очевидно, что реляционная модель — это нечто большее, чем просто "таблицы плюс операции сокращения, проекции и соединения", хотя ее неформально довольно
	часто характеризуют именно таким образом. 
	
	\item \textbf{Постреляционные}. \newline
	[Ссылка на Маркина]
	В связи с быстрым ростом количества данных и их усложнением
	возникла необходимость в поиске новых подходов к хранению и обработке, отличных от реляционных. Таким решением стала NoSQL-технология.
	
	Постреляционная модель является расширением реляционной
	модели. Она снимает ограничение неделимости данных, допуская
	многозначные поля, значения которых состоят из подзначений,
	и набор значений воспринимается как самостоятельная таблица,
	встроенная в главную таблицу.
	
\end{enumerate}



\subsubsection{Выбор модели БД}





\pagebreak
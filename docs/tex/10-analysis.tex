\section{Аналитическая часть}

В данном разделе будет приведена постановка задачи, описана модель и структура базы данных, а также будут определены роли пользователей в системе.

\subsection{Метод извлечения многокомпонентных терминов на основе \\структурных моделей}

Предлагаемый метод автоматического извлечения русскоязычных многокомпонентных терминов на основе базы данных структурных моделей терминологических словосочетаний состоит из пяти основных этапов \cite{My_article_2021}.

\begin{enumerate}[label*=\arabic*.]
	\item Анализ предложения по частям речи.
	\item Удаление частей речи и их сочетаний, которые не входят в состав терминологических словосочетаний:
	
	\begin{itemize}[label*=---]
		\item глаголы;
		\item союзы;
		\item местоимения;
		\item частицы;
		\item знаки препинания;
		\item <<наречие + предлог>>.
		
	\end{itemize}
	
	\item Удаление из оставшихся терминов-кандидатов стоп-слов, указанных в специальной зоне словаря. Под стоп-словами понимаются слова, которые образуют широко используемые коллокации с терминами, но в совокупности не являются терминами по сути, например <<современная химия>>, <<рассматриваемый метод синтеза о-гликозидов>>.
	\item Соотнесение полученных цепочек слов с шаблонами терминологических словосочетаний, которые хранятся в базе структурных моделей терминов.
	\item Проверка полученных терминов-кандидатов по словарю корпуса.
	
	\begin{enumerate}[label*=\arabic*.]
		
	\item Если термин-кандидат есть в словаре, то он извлекается как термин.
	\item Если полученный термин-кандидат отсутствует в словаре, то он отправляется терминологу для ручной обработки.
	\item Если термин-кандидат состоит из нескольких слов, то производится попытка разбить его на несколько терминов.
	
	\end{enumerate}
	
\end{enumerate}



\subsection{Постановка задачи}

Диаграмма, оформленная в соответствии с нотацией IDEF0 и отражающая декомпозицию алгоритма работы системы извлечения многокомпонентных терминов, представлена в приложении А. Исходя из специфики решаемой задачи, можно сформулировать требования к приложению \cite{My_article_2022}.


\begin{enumerate}[label*=\arabic*.]
	\item Задание исходных текстов должно производиться из файла или через специально предусмотренные поля ввода.
	\item Пользователю должна быть предоставлена возможность перед сохранением терминов в базу данных выполнять над ними следующие операции.
	
	\begin{enumerate}[label*=\arabic*.]
		\item Редактирование.
		\item Сопоставление переводных эквивалентов.
		\item Присвоение характеристик.
		
	\end{enumerate}

	\item Пользователь может анализировать и редактировать добавленные в базу данных термины путём их поиска по заданным характеристикам.
	
\end{enumerate}



\subsection{Формализация ролей}

Для работы с системой обязательным этапом является прохождение аутентификации. Пользователь может работать в системе под одной из следующих ролей.

\begin{enumerate}[label*=\arabic*.]
	\item Студент --- пользователь, имеющий возможность обрабатывать тексты, сохранять выделенные термины, а также анализировать и редактировать только те термины, которые он выделил.
	\item Преподаватель --- пользователь, обладающий функционалом, доступным роли <<Студент>>, а также имеющий возможность анализировать и редактировать все термины, хранящиеся в базе данных.
	\item Администратор --- пользователь, имеющий возможности, доступные роли <<Преподаватель>>, а также имеющий в распоряжении специальные функции для анализа состояния системы и настройки её работы. Также администратор имеет возможность создавать аккаунты, соответствующие любой из ролей в системе.
	
\end{enumerate}

В ходе использования приложения должна быть предусмотрена возможность смены ролей путём прохождения повторной аутентификации.

Также стоит отметить, что до входа в аккаунт пользователь считается неавторизованным и не имеет доступа к функционалу системы, так как любая работа с терминами должна быть персонализирована.

На рисунке \ref{fig:use-case} представлена диаграмма вариантов использования системы в соответствии с выделенными типами пользователей.

\begin{figure}[h]
	\centering
	\includegraphics[width=\textwidth ]{img/Use-case/Use-case.drawio.png}
	\caption{Диаграмма вариантов использования}
	\label{fig:use-case}
\end{figure} 

\clearpage



\subsection{Формализация данных}

Разрабатываемая база данных должна хранить информацию о следующих сущностях:

\begin{itemize}[label*=---]
	\item пользователи;
	\item русскоязычные термины;
	\item англоязычные термины;
	\item характеристики терминов;
	\item структурные модели терминов;
	\item части речи;
	\item контексты употребления терминов.
	
\end{itemize}

На рисунке \ref{fig:er} представлена ER-диаграмма сущностей в нотации Чена, показывающая сущности, их атрибуты и связи между сущностями в разрабатываемой базе данных.

\begin{figure}[h]
	\centering
	\includegraphics[width=\textwidth ]{img/ER/ER.drawio.png}
	\caption{ER-диаграмма сущностей базы данных}
	\label{fig:er}
\end{figure} 

\clearpage



\subsection{Анализ моделей БД}

\subsubsection{Классификация СУБД по модели данных}

По модели данных СУБД можно разделить на следующие типы.

\begin{enumerate}[label*=\arabic*.]
	\item \textbf{Дореляционные}. \newline
	Старые (дореляционные) системы можно разделить на три большие категории: системы с инвертированными списками (inverted list), иерархические (hierarchic) и сетевые (network) \cite{Date_old}. 
	
	% В настоящей книге эти категории подробно не рассматриваются, поскольку, по крайней мере, с точки зрения технологии, их можно считать устаревшими.
		
	\begin{enumerate}[label*=\arabic*.]
		\item \textbf{Инвертированные списки (файлы)}. \newline
		БД на основе инвертированных списков представляет собой совокупность файлов (таблиц), содержащих записи. Для записей в файле определен некоторый порядок, диктуемый физической организацией данных. Для каждого файла может быть определено произвольное число других упорядочений на основании значений некоторых полей записей (инвертированных списков). Обычно для этого используются индексы. В такой модели данных отсутствуют ограничения целостности как таковые. Все ограничения на возможные экземпляры БД задаются теми программами, которые работают с БД. Одно из немногих ограничений, которое может присутствовать --- это ограничение, задаваемое уникальным индексом. 
		
		\item \textbf{Иерархические}. \newline
		Иерархическая модель БД состоит из объектов с указателями от родительских объектов к дочерним, соединяя вместе связанную информацию. Иерархические БД могут быть представлены в виде дерева. Их производительность в значительной степени зависит от подхода, выбранного самим пользователем (прикладным программистом и/или администратором базы данных).
		
		\item \textbf{Сетевые}. \newline
		К основным понятиям сетевой модели БД относятся элемент (узел) и связь. Узел --- это совокупность атрибутов данных, описывающих некоторый объект. Сетевые БД могут быть представлены в виде графа. В сетевой БД логика процедуры выборки данных зависит от физической организации этих данных. Поэтому эта модель не является полностью независимой от приложения. Другими словами, если необходимо изменить структуру данных, то нужно изменить и приложение.
		
	\end{enumerate}
	
	\item \textbf{Реляционные}. \newline
	Ключевые особенности реляционной модели данных \cite{Date_old}:
	
	\begin{itemize}[label*=---]
		\item  Определение реляционной системы требует, чтобы база данных только воспринималась пользователем как набор таблиц. Таблицы в реляционной системе являются логическими, а не физическими структурами. Таблицы представляют собой абстракцию способа физического хранения данных, в которой детали реализации на уровне физической памяти скрыты от пользователя.
		\item  Информационный принцип: все информационное наполнение базы данных представлено одним и только одним способом, а именно --- явным заданием значений, помещенных в позиции столбцов в строках таблицы. Этот метод представления --- единственно возможный для реляционных баз данных. В частности, нет никаких указателей, связывающих одну таблицу с другой.
		
	\end{itemize}
	
	Реляционная модель состоит из пяти компонентов \cite{Date_old}.
	
	\begin{itemize}[label*=---]
		\item  Неограниченный набор скалярных типов (включая, в частности, логический тип).
		\item  Генератор типов отношений и соответствующая интерпретация для сгенерирован ных типов отношений.
		\item  Возможность определения переменных отношения для указанных сгенерированных типов отношений.
		\item  Операция реляционного присваивания для присваивания реляционных значений указанным переменным отношения.
		\item  Неограниченный набор общих реляционных операторов (реляционная алгебра) для получения значений отношений из других значений отношений.
		
	\end{itemize}
	
	% Вполне очевидно, что реляционная модель --- это нечто большее, чем просто "таблицы плюс операции сокращения, проекции и соединения", хотя ее неформально довольно часто характеризуют именно таким образом. 
	
	\item \textbf{Постреляционные}. \newline
	В связи с быстрым ростом количества данных и их усложнением возникла необходимость в поиске новых подходов к хранению и обработке, отличных от реляционных. Таким решением стала NoSQL-технология (Not Only SQL). Постреляционная модель является расширением реляционной модели. Она снимает ограничение неделимости данных, допуская многозначные поля, значения которых не являются атомарными, и набор значений воспринимается как самостоятельная таблица, встроенная в главную таблицу \cite{Markin}.
	
	Наиболее популярными типами РаБД являются:
	
	\begin{itemize}[label*=---]
		\item ключ-значение (Redis, Tarantool, Oracle NoSQL DB);
		\item колоночные (Vertica, ClickHouse, HBase);
		\item документо-ориентированные (CouchDB, MongoDB);
		\item графовые (InfoGrid, GraphX, Neo4j).
	
	\end{itemize}
	
\end{enumerate}



\subsubsection{Выбор модели БД}

Для долговременного хранения данных будет использоваться реляционная модель по следующим причинам:

\begin{enumerate}[label*=\arabic*)]
	\item данные имеют чётко заданную структуру;
	\item исключается дублирование данных за счёт использования связей между отношениями с помощью внешних ключей;
	\item доступ к данным отделяется от способа их организации на уровне физической памяти;
	
\end{enumerate}

Для кратковременного хранения данных о текущей сессии пользователя (в частности, сохранённых им терминов) будет использоваться нереляционная модель по следующим причинам:

\begin{enumerate}[label*=\arabic*)]
	\item данные могут не иметь общей структуры;
	\item появляется возможность хранения вложенных структур данных;
	\item повышается быстродействие за счёт возможности хранения всех данных в оперативной памяти (In-Memory DataBase);
	
\end{enumerate}

In-Memory --- набор концепций хранения данных, основанных на их сохранении в оперативной памяти сервера и использовании диска для хранения резервных копий. Быстродействие In-Memory баз данных по сравнению с реляционными позволит повысить отзывчивость системы, уменьшив время ожидания пользователя при выполнении сохранения и поиска терминов.



\pagebreak
\section*{Список использованных источников}
\addcontentsline{toc}{section}{Список использованных источников}

\begingroup
\renewcommand{\section}[2]{}
\begin{thebibliography}{}
	
	\bibitem{intro_1}
	Когаловский М.Р. Энциклопедия технологий баз данных. – Москва: Финансы и статистика, 2002. – 800 с.
	
	\bibitem{intro_2}
	Алферова Т.К., Леонов А.В., Филиппов К.В. О состоянии автоматизированной базы данных терминов и определений в области ОП // Компетентность. – 2014.– № 7(118). – С. 10-14.
	
	\bibitem{intro_3}
	Горбач Т.А., Грибова В.В., Окунь Д.Б., Петряева М.В., Шалфеева Е.А., Шахгельдян К.И. База терминов нейрохирургии для интеллектуальной обработки биомедицинских данных // Сборник материалов XIII международной научной конференции: «Системный анализ в медицине». – Благовещенск, 2019. – С. 82-85.
	
	\bibitem{intro_4}
	Кузнецов И.О. Автоматическое извлечение двусловных терминов по тематике «Нанотехнологии в медицине» на основе корпусных данных // Научно-техническая информация. Сер. 2. – 2013. – № 5. – С. 25-33.
	
	\bibitem{intro_5}
	Becerro F. B. Phraseological variations in medical-pharmaceutical terminology and its applications for English and German into Spanish translations // SciMedicine Journal. – 2020. – № 2(1). – Р.22-29. DOI: 10.28991/SciMedJ-2020-0201-4.
	
	\bibitem{intro_6}
	Simon N. I., Kešelj V. August. Automatic term extraction in technical domain using part-of-speech and common-word features // Proceedings of the ACM Symposium on Document Engineering. – 2018. – P. 1-4. DOI:10.1145/3209280.3229100.
	
	\bibitem{intro_7}
	Клышинский Э.С., Кочеткова Н.А., Карпик О.В. Метод выделения коллокаций с использованием степенного показателя в распределении Ципфа // Новые информационные технологии в автоматизированных системах. – 2018. – № 21. – С. 220-225.
	
	\bibitem{intro_8}
	Кочеткова Н.А. Метод извлечения технических терминов с использованием усовершенствованной меры странности // Научно-техническая информация. Сер. 2.– 2015. –№ 5. – С. 25-32;
	
	Kochetkova N.A. A Method for Extracting Technical Terms Using the Modified Weirdness Measure // Automatic Documentation and Mathematical Linguistics. – 2015 – Vol. 49, № 3. – Р. 89-95.
	
	\bibitem{intro_9}
	Захаров В.П., Хохлова М.В. Автоматическое извлечение терминов из специальных текстов с использованием дистрибутивно-статистического метода как инструмент создания тезаурусов // Структурная и прикладная лингвистика. – 2012. – № 9. – С. 222-233.
	
	\bibitem{intro_10}
	Terryn A., Hoste V., Lefever E. In no uncertain terms: a dataset for monolingual and multilingual automatic term extraction from comparable corpora // Language Resources and Evaluation. – 2020. – Vol. 54, № 2. – P. 385-418. DOI:10.1007/ s10579-019-09453-9.
	
	\bibitem{My_article_2021}
	Бутенко Ю.И. Строганов Ю.В., Сапожков А.М. Метод извлечения русскоязычных многокомпонентных терминов в корпусе научно-технических текстов // Прикладная информатика. – 2021. – № 6. – С. 21-27. DOI: 10.37791/2687-0649-2021-16-6-21-27.
	
	\bibitem{My_article_2022}
	Бутенко Ю.И. Строганов Ю.В., Сапожков А.М. Система извлечения многокомпонентных терминов и их переводных эквивалентов из параллельных научно-технических текстов // НТИ. Сер. 2. ИНФОРМ. ПРОЦЕССЫ И СИСТЕМЫ. – 2022. – № 9. – С. 12-21. ISSN 0548-0027.
	
	\bibitem{Date_new} К. Дж. Дейт
	SQL и реляционная теория. Как грамотно писать код на SQL. – Пер. с англ. – СПб.: Символ-Плюс, 2010. – 480 с., ил. ISBN 978-5-93286-173-8
	
	\bibitem{Date_old}
	Дейт, К. Дж. Введение в системы баз данных, 8-е издание.: Пер. с англ. — М.: Издательский дом <<Вильяме>>, 2005. — 1328 с.: ил. — Парал. тит. англ. ISBN 5-8459-0788-8 (рус.)
	
	\bibitem{Markin}
	Маркин, А. В. Системы графовых баз данных. Neo4j : учебное пособие для вузов. — Москва : Издательство Юрайт, 2021. — 178 с. — (Высшее образование). ISBN 978-5-534-13996-9
	
	\bibitem{microservices}
	Ньюмен, С. Создание микросервисов, 2-е издание.: Пер. с англ. — СПб.: Издательство <<Питер>>, 2023. — 624 с.: ил. ISBN 978-5-4461-1145-9
	
	\bibitem{gRPC}
	gRPC: A high performance, open source universal RPC framework [Электронный ресурс]. — Режим доступа: \url{https://grpc.io/} (дата обращения: 27.03.2023).
	
	\bibitem{Swagger}
	Swagger: API Documentation \& Design Tools for Teams [Электронный ресурс]. — Режим доступа: \url{https://swagger.io/} (дата обращения: 15.04.2023).
	
	\bibitem{Fielding}
	Roy Thomas Fielding Architectural Styles and the Design of Network-based Software Architectures [Электронный ресурс]. — Режим доступа: \url{https://www.ics.uci.edu/~fielding/pubs/dissertation/top.htm} (дата обращения: 27.03.2023).
	
	\bibitem{time_series}
	Иванова Е.В., Цымблер М.Л. Обзор современных систем обработки временных рядов // Вестник ЮУрГУ. Серия: Вычислительная математика и информатика. 2020. Т. 9, № 4. С. 79–97. DOI: 10.14529/cmse200406.
	
	\bibitem{patterns}
	Гамма Э., Хелм Р., Джонсон Р., Влиссидес Дж. Паттерны объектно-ориентированного проектирования. — СПб.: Питер, 2020. — 448 с.: ил. — (Серия «Библиотека программиста»). ISBN 978-5-4461-1595-2
	
	\bibitem{repository_pattern}
	The Repository Pattern Explained [Электронный ресурс]. — Режим доступа: \url{https://blog.sapiensworks.com/post/2014/06/02/The-Repository-Pattern-For-Dummies.aspx} (дата обращения: 06.02.2023).
	
	\bibitem{dependency_injection}
	Seemann, M. and van Deursen, S. Dependency Injection. Principles, Practices, and Patterns. — Manning, 2019. — 552 с. ISBN 978-1-6172-9473-0
	
	\bibitem{PostgreSQL}
	PostgreSQL: The World's Most Advanced Open Source Relational Database [Электронный ресурс]. — Режим доступа: \url{https://www.postgresql.org/} (дата обращения: 15.04.2023).
	
	\bibitem{fstec}
	Государственный реестр сертифицированных средств защиты информации [Электронный ресурс]. — Режим доступа: \url{https://fstec.ru/tekhnicheskaya-zashchita-informatsii/dokumenty-po-sertifikatsii/153-sistema-sertifikatsii/591-gosudarstvennyj-reestr-sertifitsirovannykh-sredstv-zashchity-informatsii-n-ross-ru-0001-01bi00} (дата обращения: 15.04.2023).
	
	\bibitem{Redis}
	Redis. The open source, in-memory data store used by millions of developers as a database, cache, streaming engine, and message broker. [Электронный ресурс]. — Режим доступа: \url{https://redis.io/} (дата обращения: 15.04.2023).
	
	\bibitem{InfluxDB}
	InfluxData: InfluxDB Times Series Data Platform [Электронный ресурс]. — Режим доступа: \url{https://www.influxdata.com/} (дата обращения: 15.04.2023).
	
	\bibitem{log_tree}
	 O’Neil P., Cheng E., Gawlick D., O’Neil E. The log-structured merge-tree (LSM-tree) // Acta Informatica. 1996. Vol. 33. P. 351–385.
	 
	 \bibitem{InfluxQL}
	 InfluxDB Query Language [Электронный ресурс]. — Режим доступа: \url{https://docs.influxdata.com/influxdb/v1.3/query_language/} (дата обращения: 15.04.2023).
	 
	 \bibitem{Golang}
	 The Go Programming Language [Электронный ресурс]. — Режим доступа: \url{https://go.dev/} (дата обращения: 15.04.2023).
	 
	 \bibitem{Gin}
	 Gin Web Framework [Электронный ресурс]. — Режим доступа: \url{https://gin-gonic.com/} (дата обращения: 15.04.2023).
	 
	 \bibitem{openAPI}
	 OpenAPI Specification - Version 3.0.3 - Swagger [Электронный ресурс]. — Режим доступа: \url{https://swagger.io/specification/} (дата обращения: 15.04.2023).
	 
	 \bibitem{Docker}
	 Docker: Accelerated, Containerized Application Development [Электронный ресурс]. — Режим доступа: \url{https://www.docker.com/} (дата обращения: 15.04.2023).
	 
	 \bibitem{Docker_compose}
	 Docker Compose overview [Электронный ресурс]. — Режим доступа: \url{https://docs.docker.com/compose/} (дата обращения: 15.04.2023).
\end{thebibliography}
\endgroup

\pagebreak
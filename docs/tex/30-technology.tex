\section{Технологическая часть}

% В данном разделе будут описаны критерии сравнения операционных систем для устройств интернета вещей.

\subsection{Выбор СУБД}

% PostgreSQL --- основное хранилище, Redis --- кеш данных текущей сессии (по окончании сессии перекидывается на диск на стороне сервера (после возврата из функции обработки подключения, т.е. на стороне сервера надо разделять (идентифицировать) конкретные сессии (не пользователей!))).

Для приложения необходимо выбрать СУБД, которые будут использоваться для долговременного хранения данных и кеширования данных пользовательских сессий.

В качестве основного хранилища данных была выбрана объектно-реляционная СУБД PostgreSQL, которая позволит реализовать многослойную структуру хранения данных. Также стоит отметить, что PostgreSQL сертифицирована ФСТЭК России и имеет открытый исходный код.
% поддерживает функции поддержки безопасности и другие многочисленные расширения

Для хранения сессий было выбрано in-memory хранилище Redis. Данная СУБД удовлетворяет необходимым требованиям (хранилище типа ключ-значение) и не требует дополнительной настройки для начала использования.

Также в целях повышения безопасности системы необходимо выбрать СУБД для хранения логов. Для данной цели была выбрана СУБД InfluxDB. В отличие от PostgreSQL, InfluxDB более компактно хранит данные и оптимизирована на запись данных, что подходит для организации сервера логов.



\subsection{Средства реализации [TODO]}

Система, взаимодействующая с базой данных, представляет из себя веб-сервер, доступ к которому осуществляется с помощью REST API. Для реализации будет использоваться язык программирования Golang. Этот язык создан для разработки микросервисных веб приложений. В данном случае приложение будет состоять из двух микросервисов: сервис сессий и сервис бизнес-логики. 

Для взаимодействия с базами данных будут использоваться драйвера, написанные для языка golang, предоставляющие интерфейс взаимодействия посредством языка программирования.

Для реализации REST API будет использоваться веб фреймворк echo [11]. Документирование REST API будет осуществляться с помощью Swagger, который поддерживает протокол openAPI.

Для сборки приложения в готовый продукт был выбран оркестратор Docker контейнеров docker-compose.

Docker позволяет изолировать приложение и разворачивать его на любой машине, независимо от установленных зависимостей. Это реализуется благодаря наличию всех требуемых зависимостей внутри контейнера.

В отличии от виртуальных машин, имеющих хост-ОС и гостевую ОС. Контейнеры размещаются на одном физическом сервере с операционной системой хоста, которая разделяет их между собой. Совместное использование ОС хоста между контейнерами делает их менее требовательными к мощности компьютера.

Docker-compose связывает контейнеры в одну систему - в отдельных контейнерах будет сообираться два микросервиса приложения, postgresql и redis.

Внешние средства для мониторинга состояния системы: Docker Desktop (или UI сервиса, на котором развёрнуто приложение), Grafana, PGAdmin.



\subsection{Детали реализации [TODO]}

\subsubsection{Хранимые процедуры БД}

Реализация хранимой процедуры (инициализация + вызов).

\subsubsection{Роли БД}

Реализация ролевой модели (инициализация ролей и выдача им прав).

\subsubsection{Интерфейс приложения}

Описать API бекенда.

\subsection{Развёртывание приложения}

Инструкция по деплою + текст файла docker-compose.

\subsection{Примеры работы ПО}

Картинки.

\pagebreak
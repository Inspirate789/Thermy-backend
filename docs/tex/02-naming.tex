\section*{Определения, обозначения и сокращения}

Единица языка --- элемент системы языка, неразложимый в рамках определённого уровня членения текста и противопоставленный другим единицам в подсистеме языка, соответствующей этому уровню.

Терминологическая единица --- элемент терминологической системы, которая является языковым выражением системы понятий определенной предметной области.

Термин --- слово или словосочетание, являющееся названием определённого понятия в определенной предметной области.

Модель данных --- это абстрактное, независимое, логическое определение структур данных, операторов над данными и прочего, что в совокупности составляет абстрактную систему, с которой взаимодействует пользователь.

База данных (БД) --- совокупность взаимосвязанных данных некоторой предметной области, хранимых в памяти ЭВМ и организованных таким образом, что эти данные могут быть использованы для решения многих задач многими пользователями.
% \cite{Date_new}

% это некоторый набор перманентных (постоянно хранимых) данных, используемых прикладными программными системами какого-либо предприятия. \cite{Date_old}

Система управления базами данных (СУБД) --- приложение, обеспечивающее создание, хранение, обновление и поиск информации в базах данных.
% Лекции

Индекс --- объект базы данных, создаваемый с целью повышения производительности поиска данных.

Бизнес-сценарий --- сценарий взаимодействия пользователя с программным продуктом для достижения конкретной цели.

Сервис --- абстракция, определяющая что-то, что предоставляет услугу.

Компонент --- единица развёртывания ПО, которая является реализацией сервиса, содержащей поведение.
% \cite{dependency_injection}

Микросервис --- сервис, отвечающий за один элемент логики в определенной предметной области.
% \cite{dependency_injection}

\pagebreak

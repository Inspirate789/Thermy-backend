%----------------------- Преамбула -----------------------
\documentclass[ut8x, 14pt, oneside, a4paper]{article}

\usepackage{extsizes} % Для добавления в параметры класса документа 14pt

% Для работы с несколькими языками и шрифтом Times New Roman по-умолчанию
\usepackage[english,russian]{babel}
\usepackage{fontspec}
\setmainfont{Times New Roman}

% ГОСТовские настройки для полей и абзацев
\usepackage[left=30mm,right=10mm,top=20mm,bottom=20mm]{geometry}
\usepackage{misccorr}
\usepackage{indentfirst}
\usepackage{enumitem}
\setlength{\parindent}{1.25cm}
%\setlength{\parskip}{1em} % поменять
%\linespread{1.3}
\renewcommand{\baselinestretch}{1.5}
\setlist{nolistsep} % Отсутствие отступов между элементами \enumerate и \itemize

% Дополнительное окружения для подписей
\usepackage{array}
\newenvironment{signstabular}[1][1]{
	\renewcommand*{\arraystretch}{#1}
	\tabular
}{
	\endtabular
}

% Переопределение стандартных \section, \subsection, \subsubsection по ГОСТу;
% Переопределение их отступов до и после для 1.5 интервала во всем документе
\usepackage{titlesec}

\titleformat{\section}[block]
{\bfseries\normalsize\filcenter}{\thesection}{1em}{}

\titleformat{\subsection}[hang]
{\bfseries\normalsize}{\thesubsection}{1em}{}
\titlespacing\subsection{\parindent}{\parskip}{\parskip}

\titleformat{\subsubsection}[hang]
{\bfseries\normalsize}{\thesubsubsection}{1em}{}
\titlespacing\subsubsection{\parindent}{\parskip}{\parskip}

\newcommand{\subsubsubsection}[1]{\paragraph{#1}\mbox{}}
\setcounter{secnumdepth}{4}
\setcounter{tocdepth}{4}

\titlespacing\paragraph{\parindent}{\parskip}{\parskip}

% Работа с изображениями и таблицами; переопределение названий по ГОСТу
\usepackage{caption}
\captionsetup[figure]{name={Рисунок},labelsep=endash}
\captionsetup[table]{singlelinecheck=false, labelsep=endash}

\usepackage{graphicx}
\usepackage{slashbox} % Диагональное разделение первой ячейки в таблицах

% Цвета для гиперссылок и листингов
\usepackage{color}

% Гиперссылки \toc с кликабельностью
\usepackage{hyperref}
\usepackage[toc,page]{appendix}

\hypersetup{
	linktoc=all,
	linkcolor=black,
	urlcolor=black,
	colorlinks=true,
}

% Листинги
\setsansfont{Arial}
\setmonofont{Courier New}

\usepackage{xcolor} % Цвета для гиперссылок и листингов
\definecolor{comment}{rgb}{0,0.5,0}
\definecolor{plain}{rgb}{0.2,0.2,0.2}
\definecolor{string}{rgb}{0.91,0.45,0.32}
%\hypersetup{citecolor=blue}
\hypersetup{citecolor=black}

\usepackage{listings}
\lstset{
	language={SQL},
	backgroundcolor=\color{white},
	basicstyle=\fontsize{12}{14}\ttfamily, % \footnotesize
	keywordstyle=\color{blue},
	stringstyle=\color{red},
	commentstyle=\color{gray},
	numbers=left,
	numberstyle=\footnotesize,
	stepnumber=1,
	numbersep=5pt,
	frame=single,
	tabsize=4,
	captionpos=t,
	breaklines=true
}

\captionsetup[lstlisting]{format=hang, justification=justified}

\usepackage{listingsutf8}

\lstdefinelanguage{docker}{
	keywords={FROM, RUN, COPY, ADD, ENTRYPOINT, CMD,  ENV, ARG, WORKDIR, EXPOSE, LABEL, USER, VOLUME, STOPSIGNAL, ONBUILD, MAINTAINER},
	keywordstyle=\color{blue}\bfseries,
	identifierstyle=\color{black},
	sensitive=false,
	comment=[l]{\#},
	commentstyle=\color{purple}\ttfamily,
	stringstyle=\color{red}\ttfamily,
	morestring=[b]',
	morestring=[b]"
}

\lstdefinelanguage{docker-compose}{
	keywords={image, environment, ports, container_name, ports, volumes, links},
	keywordstyle=\color{blue}\bfseries,
	identifierstyle=\color{black},
	sensitive=false,
	comment=[l]{\#},
	commentstyle=\color{purple}\ttfamily,
	stringstyle=\color{red}\ttfamily,
	morestring=[b]',
	morestring=[b]"
}

\lstdefinelanguage{docker-compose-3}{
	keywords={version, volumes, services, networks},
	keywordstyle=\color{blue}\bfseries,
	keywords=[2]{image, environment, ports, container_name, ports, links, build, restart, healthcheck, depends_on, command, volumes, networks},
	keywordstyle=[2]\color{olive}\bfseries,
	identifierstyle=\color{black},
	sensitive=false,
	comment=[l]{\#},
	commentstyle=\color{purple}\ttfamily,
	stringstyle=\color{red}\ttfamily,
	morestring=[b]',
	morestring=[b]"
}



\newcommand\YAMLcolonstyle{\color{red}\mdseries}
\newcommand\YAMLkeystyle{\color{black}\bfseries}
\newcommand\YAMLvaluestyle{\color{blue}\mdseries}

\makeatletter

% here is a macro expanding to the name of the language
% (handy if you decide to change it further down the road)
\newcommand\language@yaml{yaml}

\expandafter\expandafter\expandafter\lstdefinelanguage
\expandafter{\language@yaml}
{
	keywords={true,false,null,y,n},
	keywordstyle=\color{darkgray}\bfseries,
	basicstyle=\YAMLkeystyle,                                 % assuming a key comes first
	sensitive=false,
	comment=[l]{\#},
	morecomment=[s]{/*}{*/},
	commentstyle=\color{purple}\ttfamily,
	stringstyle=\YAMLvaluestyle\ttfamily,
	moredelim=[l][\color{orange}]{\&},
	moredelim=[l][\color{magenta}]{*},
	moredelim=**[il][\YAMLcolonstyle{:}\YAMLvaluestyle]{:},   % switch to value style at :
	morestring=[b]',
	morestring=[b]",
	literate =    {---}{{\ProcessThreeDashes}}3
	{>}{{\textcolor{red}\textgreater}}1     
	{|}{{\textcolor{red}\textbar}}1 
	{\ -\ }{{\mdseries\ -\ }}3,
}

% switch to key style at EOL
\lst@AddToHook{EveryLine}{\ifx\lst@language\language@yaml\YAMLkeystyle\fi}
\makeatother

\newcommand\ProcessThreeDashes{\llap{\color{cyan}\mdseries-{-}-}}

%\lstset{
%	basicstyle=\footnotesize\ttfamily,
%	language=SQL, % Или другой ваш язык -- см. документацию пакета
%	commentstyle=\color{comment},
%	numbers=left,
%	numberstyle=\tiny\color{plain},
%	numbersep=5pt,
%	tabsize=4,
%	extendedchars=\true,
%	breaklines=true,
%	keywordstyle=\color{blue},
%	frame=b,
%	stringstyle=\ttfamily\color{string}\ttfamily,
%	showspaces=false,
%	showtabs=false,
%	xleftmargin=17pt,
%	framexleftmargin=17pt,
%	framexrightmargin=5pt,
%	framexbottommargin=4pt,
%	showstringspaces=false,
%	inputencoding=utf8x,
%	keepspaces=true
%}
%
%\DeclareCaptionLabelSeparator{line}{\ --\ }
%\DeclareCaptionFont{white}{\color{white}}
%\DeclareCaptionFormat{listing}{\colorbox[cmyk]{0.43,0.35,0.35,0.01}{\parbox{\textwidth}{\hspace{15pt}#1#2#3}}}
%\captionsetup[lstlisting]{
%	format=listing,
%	labelfont=white,
%	textfont=white,
%	singlelinecheck=false,
%	margin=0pt,
%	font={bf,footnotesize},
%	labelsep=line
%}

\usepackage{ulem} % Нормальное нижнее подчеркивание
\usepackage{hhline} % Двойная горизонтальная линия в таблицах
\usepackage[figure,table]{totalcount} % Подсчет изображений, таблиц
\usepackage{rotating} % Поворот изображения вместе с названием
\usepackage{lastpage} % Для подсчета числа страниц

\usepackage{threeparttable}
\usepackage{csvsimple}
\usepackage{longtable,ltcaption,booktabs}
\usepackage{tabularx}
\usepackage{multirow}
\usepackage{ragged2e}

\makeatletter
\renewcommand\@biblabel[1]{#1.}
\makeatother

\usepackage{color}
\usepackage[cache=false, newfloat]{minted}
\newenvironment{code}{\captionsetup{type=listing}}{}
\SetupFloatingEnvironment{listing}{name=Листинг}

\usepackage{amsmath}
\usepackage{slashbox}

\usepackage{multirow} %for multirow
\usepackage{rotating} %for sideways, rotating "samples" at 90°"
\usepackage{bigstrut} %for bigstrut



% Для ссылок в списке использованных источников и возможности их переноса
\usepackage{url}
\def\UrlBreaks{\do\A\do\B\do\C\do\D\do\E\do\F\do\G\do\H\do\I\do\J
	\do\K\do\L\do\M\do\N\do\O\do\P\do\Q\do\R\do\S\do\T\do\U\do\V
	\do\W\do\X\do\Y\do\Z\do\[\do\\\do\]\do\^\do\_\do\`\do\a\do\b
	\do\c\do\d\do\e\do\f\do\g\do\h\do\i\do\j\do\k\do\l\do\m\do\n
	\do\o\do\p\do\q\do\r\do\s\do\t\do\u\do\v\do\w\do\x\do\y\do\z
	\do\.\do\@\do\\\do\/\do\!\do\_\do\|\do\;\do\>\do\]\do\)\do\,
	\do\?\do\'\do+\do\=\do\#}
